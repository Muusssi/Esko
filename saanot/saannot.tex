\documentclass[a4paper, 12pt, finnish]{scrartcl}
\usepackage{babel}
\usepackage[utf8]{inputenc}
\usepackage[T1]{fontenc}

\title{Yhdistyksen säännöt}
\author{Espoon Kopsu Frisbeegolf \& Ultimate ry}
\date{\today}

\begin{document}
\maketitle
\newpage

\tableofcontents
\newpage

\section{Yhdistyksen nimi ja kotipaikka}
Yhdistyksen nimi on Espoon Kopsu Frisbeegolf \& Ultimate ry.
Yhdistyksen kotipaikka on Espoo.

\section{Tarkoitus ja toiminnan laatu}
Yhdistyksen tarkoituksena on edistää ja kehittää frisbeelajien harrastusta ja harrastusmahdollisuuksia yhdistyksen kotipaikkakunnalla ja muualla pääkaupunkiseudulla.
Yhdistyksen tarkoituksena on myös kohottaa jäsentensä fyysistä kuntoa ja terveyttä sekä herättää heissä oikeaa urheiluhenkeä.

\subsection*{Tarkoituksensa toteuttamiseksi yhdistys:}
\begin{itemize}
  \item järjestää frisbeegolf- ja ultimate-harjoituksia ja info- ja tutustumistilaisuuksia lajeihin liittyen
  \item hankkii lajeissa tarvittavia välineitä
  \item järjestää kilpailuita, retkiä, tilaisuuksia ja muuta vastaavaa toimintaa myös yhteistyössä muiden frisbeegolf- ja ultimateseurojen kanssa
  \item vaikuttaa kotipaikkakunnallaan liikuntakasvatukseen ja -suunnitteluun frisbee-lajien osalta
\end{itemize}

Toimintansa tukemiseksi yhdistys voi harjoittaa julkaisutoimintaa ja voi pitää yllä kotisivuja,
voi ottaa vastaan lahjoituksia ja testamentteja,
voi toimeenpanna rahankeräyksiä ja arpajaisia hankittuaan asianmukaisen luvan, sekä omistaa toimintaansa varten tarpeellista irtainta ja kiinteää omaisuutta.

\section{Jäsenet}
Yhdistykseen varsinaiseksi jäseneksi voidaan hyväksyä henkilö, joka hyväksyy yhdistyksen tarkoituksen ja säännöt sekä maksaa vuotuisen jäsenmaksun.
Jäseneksi liittyminen tapahtuu, kun hallitus hyväksyy liittymisen ja henkilö maksaa vuotuisen jäsenmaksun yhdistyksen kassaan tai yhdistyksen pankkitilille.
Kannattavaksi jäseneksi voidaan hyväksyä yksityinen henkilö tai oikeuskelpoinen yhteisö, joka haluaa tukea yhdistyksen tarkoitusta ja toimintaa.
Kannattavaksi jäseneksi liittyminen tapahtuu, kun hallitus hyväksyy liittymisen ja henkilö maksaa kannatusjäsenmaksun yhdityksen kassaan tai yhdityksen pankkitilille.
Kunniapuheenjohtajaksi tai kunniajäseneksi voidaan hallituksen esityksestä yhdistyksen kokouksessa kutsua henkilö, joka on huomattavasti edistänyt ja tukenut yhdistyksen toimintaa.

Jäsen voi kuulua myös muihin vastaaviin yhdistyksiin.

Jäsenellä on oikeus erota seurasta ilmoittamalla siitä kirjallisesti hallitukselle tai sen puheenjohtajalle taikka ilmoittamalla erosta seuran kokouksessa pöytäkirjaan merkitsemistä varten. Eroaminen tulee voimaan sen kalenterivuoden viimeisenä päivänä, jolloin eroilmoitus on tehty.

Yhdistyksen jäsen voidaan erottaa, jos hän on laiminlyönyt jäsenmaksun suorittamisen saamansa muistutuksen jälkeen tai jos hän toimii yhdistyksen tarkoitusperien vastaisesti.
Erottamisesta vastaa yhdistyksen hallitus.

Erotessaan yhdistyksen jäsenyydestä pitää jäsenen maksaa mahdolliset velkansa, joita hänelle on yhdistykselle sekä luovuttaa yhdistykselle kuuluva omaisuus.

Yhdistyksen hallitus pitää yllä lain vaatimaa jäsenrekisteriä, johon merkitään jäsenen täydellinen nimi, kotipaikkakunta ja sähköpostiosoite sekä jäsenluetteloa, johon merkitään täydellinen nimi ja kotipaikkakunta.


\section{Yhdistyksen jäsenyys muissa yhdistyksissä}
Yhdistyksen jäsenyyksistä muissa yhdistyksissä päättää yhdistyksen hallituksen kokous tai yleiskokous.
Yhdistys ja yhdistyksen jäsenet sitoutuvat noudattamaan niiden järjestöjen sääntöjä, joiden jäsenenä yhdistys on.

\section{Jäsenmaksu}
Varsinaisilta jäseniltä ja kannattavilta jäseniltä perittävän vuotuisen jäsenmaksun suuruudesta erikseen kummallekin jäsenryhmälle päättää syyskokous.
Kunniapuheenjohtaja ja kunniajäsenet eivät suorita jäsenmaksuja.
Seuran vanhoilta jäseniltä jäsenmaksu tulee olla maksettu viimeistään kuukautta kevätkokouksen jälkeen.


\section{Hallitus}
Yhdistyksen asioita hoitaa hallitus, johon kuuluu syyskokouksessa valitut puheenjohtaja ja 2-8 muuta varsinaista jäsentä sekä 0-8 varajäsentä.
Hallituksen toimikausi on kalenterivuosi.
Hallitus valitsee keskuudestaan varapuheenjohtaja sekä ottaa keskuudestaan tai ulkopuoleltaan sihteerin, rahastonhoitajan ja muut tarvittavat toimihenkilöt.
Hallitus kokoontuu puheenjohtajan tai hänen estyneenä ollessaan varapuheenjohtajan kutsusta, kun he katsovat siihen olevan aihetta tai kun vähintään puolet hallituksen jäsenistä sitä vaatii.
Hallitus on päätösvaltainen, kun vähintään puolet sen jäsenistä, puheenjohtaja tai varapuheenjohtaja mukaanluettuna on läsnä.
Äänestykset ratkaistaan ehdottomalla ääntenenemmistöllä.
Äänten mennessä tasan ratkaisee puheenjohtajan ääni, vaaleissa kuitenkin arpa.

Hallituksen on kutsuttava yhdistyksen hallituksen kokoukset koolle vähintään yhtä (1) päivää ennen kokousta sähköpostitse tai muulla sopimallaan tavalla.
Mikäli kaikki hallituksen jäsenet ovat läsnä tai poissa olevilta hallituksen jäseniltä saadaan suostumus, voidaan hallituksen kokous järjestää ilman erillistä kutsua.

Yhdistyksen hallitus voi erityisiä tehtäviä varten asettaa määräaikaisia toimikuntia tai toimihenkilöitä.
Näihin toimiin voidaan tarvittaessa nimetä myös kerhon ulkopuolinen henkilö.
Yhdistyksen kokous ja hallitus voivat antaa toimikuntien ja -henkilöiden toimintaan liittyviä sääntöjä ja määräyksiä.

Perustellusta syystä tai puheenjohtajan esityksestä yhdistyksen kokous voi vapauttaa hallituksen jäsenen ja hallituksen kokous hallituksen asettaman toimihenkilön tehtävästään kesken toimikautta, jolloin kokouskutsussa on mainittava asiasta.

\section{Yhdistyksen nimen kirjoittaminen}
Yhdistyksen nimen kirjoittaa hallituksen puheenjohtaja tai varapuheenjohtaja yhdessä hallituksen jäsenen kanssa, tai hallituksen nimenkirjoitukseen oikeuttama henkilö yksin.

\section{Tilikausi ja toiminnantarkastus}
Yhdistyksen tilikausi on kalenterivuosi.
Tilinpäätös tarvittavine asiakirjoineen ja hallituksen vuosikertomus on annettava toiminnantarkastajille/tilintarkastajille viimeistään kuukausi ennen kevätkokousta.
Toiminnantarkastajien/tilintarkastajien tulee antaa kirjallinen lausuntonsa viimeistään kaksi (2) viikkoa ennen kevätkokousta hallitukselle.

\section{Yhdistyksen kokoukset}
Yhdistyksen kokouksia ovat kevät- ja syyskokous sekä ylimääräiset kokoukset.
Yhdistyksen kokoukseen voidaan osallistua hallituksen tai yhdistyksen kokouksen niin päättäessä myös postitse taikka tietoliikenneyhteyden tai muun teknisen apuvälineen avulla kokouksen aikana tai ennen kokousta.
Yhdistys pitää vuosittain kaksi varsinaista kokousta.
Yhdistyksen kevätkokous pidetään tammi-toukokuussa ja syyskokous syys-joulukuussa hallituksen määräämänä päivänä.
Yhdistyksen kokouksissa on jokaisella varsinaisella jäsenellä, kunniapuheenjohtajalla ja kunniajäsenellä yksi ääni.
Kannattavalla jäsenellä on kokouksessa läsnäolo- ja puheoikeus.
Yhdistyksen kokouksen päätökseksi tulee, ellei säännöissä ole toisin määrätty, se mielipide, jota on kannattanut yli puolet annetuista äänistä.
Äänten mennessä tasan ratkaisee kokouksen puheenjohtajan ääni, vaaleissa kuitenkin arpa.

Yhdistyksen kokous on päätösvaltainen, kun se on sääntöjen mukaisesti koolle kutsuttu ja paikalla on vähintään kahdeksan (8) yhdistyksen jäsentä tai vähintään yksi neljäsosa (1/4) yhdistyksen varsinaisista jäsenistä.

Ylimääräinen kokous pidetään kun yhdistyksen kokous niin päättää tai kun hallitus katsoo siihen olevan aihetta tai kun vähintään kymmenesosa (1/10) yhdistyksen äänioikeutetuista jäsenistä sitä hallitukselta erityisesti ilmoitettua asiaa varten kirjallisesti vaatii.
Kokous on pidettävä kolmenkymmenen vuorokauden kuluessa siitä, kun vaatimus sen pitämisestä on esitetty hallitukselle.

\section{Yhdistyksen kokousten koollekutsuminen}
Hallituksen on kutsuttava yhdistyksen kokoukset koolle vähintään seitsemää (7) vuorokautta ennen kokousta sähköpostitse.
Kokouskutsun yhteydessä on esitettävä myös esityslista.

\section{Päätöksentekojärjestys ja vaalit}
Ellei säännöissä toisin mainita, tulee päätökseksi se mielipide, jota on kannattanut yli puolet äänestyksessä annetuista äänistä.
Puheenjohtajan ääni ratkaisee äänten mennessä tasan, vaalissa kuitenkin arpa.

Jos on useita päätösehdotuksia, noudatetaan yksinkertaista äänestysjärjestystä.
Vaali voidaan suorittaa suljettuna mikäli vähintään kaksi (2) osanottajista sitä vaatii.

Milloin vaaleissa on valittavana vain yksi henkilö, on hänen saatava vähintään puolet (1/2) annetuista äänistä.
Mikäli ensimmäisellä kierroksella kukaan ei ole saanut ehdotonta äänten enemmistöä, suoritetaan uusi vaali kahden (2) ensimmäisessä äänestyksessä eniten ääniä saaneen välillä.

\section{Varsinaiset kokoukset}

\subsection*{Yhdistyksen kevätkokouksessa käsitellään ainakin seuraavat asiat:}
\begin{enumerate}
  \item kokouksen avaus
  \item valitaan kokouksen puheenjohtaja, sihteeri ja kaksi pöytäkirjantarkastajaa, jotka toimivat tarvittaessa ääntenlaskijoina
  \item todetaan kokouksen laillisuus ja päätösvaltaisuus
  \item hyväksytään kokouksen työjärjestys
  \item esitetään tilinpäätös, vuosikertomus ja toiminnantarkastajien/tilintarkastajien lausunto
  \item päätetään tilinpäätöksen vahvistamisesta ja vastuuvapauden myöntämisestä hallitukselle ja muille vastuuvelvollisille
  \item käsitellään muut kokouskutsussa mainitut asiat.
\end{enumerate}

\subsection*{Yhdistyksen syyskokouksessa käsitellään ainakin seuraavat asiat:}
\begin{enumerate}
  \item kokouksen avaus
  \item valitaan kokouksen puheenjohtaja, sihteeri ja kaksi pöytäkirjantarkastajaa, jotka toimivat tarvittaessa ääntenlaskijoina
  \item todetaan kokouksen laillisuus ja päätösvaltaisuus
  \item hyväksytään kokouksen työjärjestys
  \item vahvistetaan toimintasuunnitelma, tulo- ja menoarvio sekä jäsenmaksujen suuruudet seuraavalle kalenterivuodelle
  \item valitaan hallituksen puheenjohtaja ja muut jäsenet
  \item valitaan vähintään yksi toiminnantarkastajaa ja henkilökohtainen varatoiminnantarkastajaa
  \item käsitellään muut kokouskutsussa mainitut asiat.
\end{enumerate}

Mikäli yhdistyksen jäsen haluaa saada jonkin asian yhdistyksen kevät- tai syyskokouksen käsiteltäväksi, on hänen ilmoitettava siitä hallitukselle niin hyvissä ajoin, että asia voidaan sisällyttää kokouskutsuun.

\section{Sääntöjen muuttaminen ja yhdistyksen purkaminen}
Päätös yhdistyksen sääntöjen muuttamisesta on tehtävä yhdistyksen hallituksen kokouksessa tai yleiskokouksessa vähintään neljän viidesosan (4/5) enemmistöllä annetuista äänistä.
Päätös yhdistyksen purkamisesta on tehtävä yhdistyksen varsinaisessa kokouksessa vähintään kolmen neljäsosan (3/4) enemmistöllä annetuista äänistä.
Kokouskutsussa on mainittava sääntöjen muuttamisesta tai yhdistyksen purkamisesta.
Yhdistyksen purkautuessa käytetään yhdistyksen varat yhdistyksen tarkoituksen edistämiseen purkamisesta päättävän kokouksen määräämällä tavalla.
Yhdistyksen tullessa lakkautetuksi käytetään sen varat samaan tarkoitukseen.

\section{Erityisiä määräyksiä}
Yhdistyksen tunnuksista, logoista ja sloganeista päättää yhdistyksen hallitus.

\end{document}